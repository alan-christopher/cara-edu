%% TODO: put together a proper Makefile. It's like one latexmk command to build and an `rm -f` to clean.
%% $ sudo apt-get install texinfo
%%   $ texi2pdf -c lecture.tex

\PassOptionsToPackage{table}{xcolor}
\documentclass[pdf]{beamer}
\mode<presentation>{\usetheme{Dresden}}
\usepackage{lmodern}
\usepackage{amsmath,textcomp,amssymb,geometry,graphicx,listings,array,color,amsthm}

\usepackage{tikz}
\usepackage{multicol}
\usetikzlibrary{shapes,snakes}
\usetikzlibrary{positioning}
\usetikzlibrary{arrows}
\usetikzlibrary{fit}

%% For on-slide alerting of nodes
\tikzstyle{alert} = [text=black, fill=blue!20, draw=black]
\setbeamercolor{alerted text}{fg=blue}
\tikzset{alerton/.code args={<#1>}{%
  \only<#1>{\pgfkeysalso{alert}} % \pgfkeysalso doesn't change the path
}}

%% utils for removing uninteresting sections from the navbar
%% https://tex.stackexchange.com/questions/317774/hide-section-from-sidebar
\makeatletter
\let\beamer@writeslidentry@miniframeson=\beamer@writeslidentry%
\def\beamer@writeslidentry@miniframesoff{%
  \expandafter\beamer@ifempty\expandafter{\beamer@framestartpage}{}% does not happen normally
  {%else
    % removed \addtocontents commands
    \clearpage\beamer@notesactions%
  }
}
\newcommand*{\miniframeson}{\let\beamer@writeslidentry=\beamer@writeslidentry@miniframeson}
\newcommand*{\miniframesoff}{\let\beamer@writeslidentry=\beamer@writeslidentry@miniframesoff}
\makeatother

%% preamble
\title{NASA CARA}
\subtitle{Air Traffic Control \emph{in Spaaaaaaaaace}}
\author{A.C.}
\date{\today}

\AtBeginSection[]
{
  \miniframesoff
  \begin{frame}{Outline}
  \tableofcontents
  \end{frame}
  \miniframeson
}

\definecolor{darkred}{rgb}{0.7,0,0}
\definecolor{darkgreen}{rgb}{0,0.5,0}
\definecolor{darkblue}{rgb}{0,0,0.5}
\definecolor{darkpurple}{rgb}{0.4, 0.0, 0.4}

%% Code font settings
\lstset{
  showstringspaces=false,
  basicstyle=\scriptsize\ttfamily,
  commentstyle=\color{darkred},
  stringstyle=\color{darkgreen},
  keywordstyle=\bfseries\color{darkpurple},
}

%%%%%%%%%%%%%%%%%%%%%%%%%%
% Start of Actual slides %
%%%%%%%%%%%%%%%%%%%%%%%%%%
\begin{document}
\begin{frame}
  \titlepage
\end{frame}

\section{CARA Mission}
\subsection{Purpose}
\begin{frame}{CARA in Theory}
  Mission Statement:
  \begin{quote}
    To take prudent measures, at reasonable cost, to enhance safety of flight,
    without placing an undue burden on mission operations
  \end{quote}
\end{frame}

\begin{frame}{CARA in Practice}
  Inputs:
  \begin{itemize}
  \item Ephemeris data from cooperating missions.
  \item Catalog of tracked earth-orbiting objects from Combined Space Operations Center (CSpOC).
  \end{itemize}

  Outputs:
  \begin{itemize}
  \item Alerts to protected missions on high interest events (HIEs).
  \item Advisories for protected missions on risk mitigatiions for HIEs.
  \end{itemize}
\end{frame}

\subsection{Complexity}
\begin{frame}{Kepler Orbits}
  \[ \ddot{R} = \ddot{R}_\text{2B} = \frac{Gm_\text{other}{||R||^3}R\]

  \begin{itemize}
  \item Solution known since Kepler and Newton.
    \begin{itemize}
    \item Must be a conic section.
    \item If closed, then ellipse.
    \end{itemize}
  \item A star holds its course and its aim\ldots returns and returns\ldots and is always the same
    \begin{itemize}
    \item \textit{Mais non}
    \end{itemize}
\end{frame}

\begin{frame}{Perturbation: Third Bodies}
  \[ \ddot{R} = \ddot{R}_\text{2B} + \ddot{R}_\text{PM}\]

  \begin{itemize}
  \item Gravity is a universal force.
  \item Lots of non-earth mass out there
    \begin{itemize}
    \item Luna
    \item Sol
    \item Uncounted others (fortunately negligible)
    \end{itemize}
  \item Particularly relevant for higher-altitude orbits.
  \end{itemize}
\end{frame}

\begin{frame}{Perturbation: Non-Sphericity}
  \[ \ddot{R} = \ddot{R}_\text{2B} + \ddot{R}_\text{PM} + \ddot{R}_\text{NS} \]
  % TODO: elaborate on non-sphericity of earth, and therefore more complicated geopotential
  % TODO: separate frame for shell theorem?
\end{frame}

\begin{frame}{Perturbation: Indirect Oblation}
  \[ \ddot{R} = \ddot{R}_\text{2B} + \ddot{R}_\text{PM} + \ddot{R}_\text{NS}  + \ddot{R}_\text{IO}\]
  % TODO: elaborate on acceleration of earth and therefore of reference frame.
\end{frame}

\begin{frame}{Perturbation: Drag}
  \[ \ddot{R} = \ddot{R}_\text{2B} + \ddot{R}_\text{PM} + \ddot{R}_\text{NS}  + \ddot{R}_\text{IO} + \ddot{R}_\text{D} \]

  \begin{itemize}
  \item Drag equation: $F_\text{D} = \frac{1}{2}\rho v^2 C_\text{D} A$
  \item Scales by
    \begin{itemize}
    \item Object shape and orientation ($C_\text{D}, A$).
    \item Square of object velocity $v^2$.
    \item Atmospheric density $rho$, drops ~exponentially with altitude.
    \end{itemize}
  \item LEO objects (altitude < 2000 km) are low and fast
    \begin{itemize}
    \item experience non-negligible drag
    \end{itemize}
  \item Bonus: non-periodic and non-conservative.
  \end{itemize}
\end{frame}

\begin{frame}{Perturbation: Solar Radiation Pressure}
  \[ \ddot{R} = \ddot{R}_\text{2B} + \ddot{R}_\text{PM} + \ddot{R}_\text{NS}  + \ddot{R}_\text{IO} + \ddot{R}_\text{D} + \ddot{R}_\text{SRP}\]
  % TODO: elaborate on the effects of SRP on a HEO satellite.
\end{frame}

\begin{frame}{Perturbation: Thrust}
  \[ \ddot{R} = \ddot{R}_\text{2B} + \ddot{R}_\text{PM} + \ddot{R}_\text{NS}  + \ddot{R}_\text{IO} + \ddot{R}_\text{D} + \ddot{R}_\text{SRP} + \ddot{R}_\text{T} \]
  % TODO: elaborate on the capabilities of satellites to maneuver
\end{frame}

\subsection{Consequence}

\begin{frame}{Mission Safety}
  % TODO: back of the envelope risk calculation for satellite cost.
  % TODO: make a caveat that 85% of conjunctions aren't currently accounted for.
\end{frame}

\begin{frame}{Kessler Syndrome}
  % TODO: go into the self-replicating nature of space-junk.
\end{frame}

\section{Conjunction Identification}
\subsection{Volumetric Screening}
% TODO: CARA's volumetric screening process.

\section{Conjunction Analysis}
\subsection{Risk Measures}

\begin{frame}{Standoff Distance}
  % TODO: summary of standoff distance as a measure of risk of collsion
\end{frame}

\begin{frame}{Probability of Collision}
  % TODO: summary of P_c as a measure of risk of collsion. Make sure to mention "probability dilution".
\end{frame}

\begin{frame}{Severity Estimation}
  % TODO: something about estimating the amount of new space trash (collision consequence).
\end{frame}

\subsection{2D $P_C$}
% TODO: describe the 2D P_C algorithm

\subsection{3D $P_C$}
% TODO: describe 3D P_C algorithm

\subsection{Monte Carlo}
% TODO: describe from-epoch and from-TCA monte carlo methods.

%% Q&A and Further reading.
\miniframesoff
\section*{}
\begin{frame}{References}[allowframebreaks]
    \tiny
    \printbibliography
\end{frame}

\begin{frame}{Self Link}
  \begin{itemize}
  \item The source for this presentation is hosted at
    \url{https://github.com/alan-christopher/cara-edu}.
  \end{itemize}
\end{frame}
\end{document}
