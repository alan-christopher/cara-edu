%% TODO: put together a proper Makefile. It's like one latexmk command to build and an `rm -f` to clean.
%% $ sudo apt-get install texinfo
%%   $ texi2pdf -c lecture.tex

\PassOptionsToPackage{table}{xcolor}
\documentclass[pdf]{beamer}
\mode<presentation>{\usetheme{Dresden}}
\usepackage{lmodern}
\usepackage{amsmath,textcomp,amssymb,geometry,graphicx,listings,array,color,amsthm}

\usepackage[noend]{algpseudocode}

\usepackage{tikz}
\usepackage{multicol}
\usetikzlibrary{shapes,snakes}
\usetikzlibrary{positioning}
\usetikzlibrary{arrows}
\usetikzlibrary{fit}

%% For on-slide alerting of nodes
\tikzstyle{alert} = [text=black, fill=blue!20, draw=black]
\setbeamercolor{alerted text}{fg=blue}
\tikzset{alerton/.code args={<#1>}{%
  \only<#1>{\pgfkeysalso{alert}} % \pgfkeysalso doesn't change the path
}}

%% utils for removing uninteresting sections from the navbar
%% https://tex.stackexchange.com/questions/317774/hide-section-from-sidebar
\makeatletter
\let\beamer@writeslidentry@miniframeson=\beamer@writeslidentry%
\def\beamer@writeslidentry@miniframesoff{%
  \expandafter\beamer@ifempty\expandafter{\beamer@framestartpage}{}% does not happen normally
  {%else
    % removed \addtocontents commands
    \clearpage\beamer@notesactions%
  }
}
\newcommand*{\miniframeson}{\let\beamer@writeslidentry=\beamer@writeslidentry@miniframeson}
\newcommand*{\miniframesoff}{\let\beamer@writeslidentry=\beamer@writeslidentry@miniframesoff}
\makeatother

%% preamble
\title{NASA CARA}
\subtitle{Air Traffic Control \emph{in Spaaaaaaaaace}}
\author{A.C.}
\date{\today}

\AtBeginSection[]
{
  \miniframesoff
  \begin{frame}{Outline}
  \tableofcontents
  \end{frame}
  \miniframeson
}

\definecolor{darkred}{rgb}{0.7,0,0}
\definecolor{darkgreen}{rgb}{0,0.5,0}
\definecolor{darkblue}{rgb}{0,0,0.5}
\definecolor{darkpurple}{rgb}{0.4, 0.0, 0.4}

%% Code font settings
\lstset{
  showstringspaces=false,
  basicstyle=\scriptsize\ttfamily,
  commentstyle=\color{darkred},
  stringstyle=\color{darkgreen},
  keywordstyle=\bfseries\color{darkpurple},
}

%%%%%%%%%%%%%%%%%%%%%%%%%%
% Start of Actual slides %
%%%%%%%%%%%%%%%%%%%%%%%%%%
\begin{document}
\begin{frame}
  \titlepage
\end{frame}

\section{CARA Mission}
\subsection{Purpose}
\begin{frame}{CARA in Theory}
  Mission Statement:
  \begin{quote}
    To take prudent measures, at reasonable cost, to enhance safety of flight,
    without placing an undue burden on mission operations
  \end{quote}
\end{frame}

\begin{frame}{CARA in Practice}
  Inputs:
  \begin{itemize}
  \item Ephemeris data from cooperating missions.
  \item Catalog of tracked earth-orbiting objects from Combined Space Operations
    Center (CSpOC).
  \end{itemize}

  Outputs:
  \begin{itemize}
  \item Alerts to protected missions on high interest events (HIEs).
  \item Advisories for protected missions on risk mitigatiions for HIEs.
    \begin{itemize}
    \item Hopefully avoid more Kosmos-Iridium incidents.
    \end{itemize}
  \end{itemize}
\end{frame}

\subsection{Complexity}
\begin{frame}{Kepler Orbits}
  \[ \ddot{R} = \ddot{R}_\text{2B} = \frac{Gm_\text{other}}{||R||^3}R \]

  \begin{itemize}
  \item Solution known since Kepler and Newton.
    \begin{itemize}
    \item Must be a conic section.
    \item If closed, then ellipse.
    \end{itemize}
  \item A star holds its course and its aim\ldots returns and returns\ldots and is always the same
    \begin{itemize}
    \item \textit{Mais non}
    \end{itemize}
  \end{itemize}
\end{frame}

\begin{frame}{Perturbation: Third Bodies}
  \[ \ddot{R} = \ddot{R}_\text{2B} + \ddot{R}_\text{PM}\]

  \begin{itemize}
  \item Gravity is a universal force.
  \item Lots of non-earth mass out there
    \begin{itemize}
    \item Luna
    \item Sol
    \item Uncounted others (fortunately negligible)
    \end{itemize}
  \item Particularly relevant for higher-altitude orbits.
  \end{itemize}
\end{frame}

% TODO: add image (maybe whole frame)
\begin{frame}{Perturbation: Non-Sphericity}
  \[ \ddot{R} = \ddot{R}_\text{2B} + \ddot{R}_\text{PM} + \ddot{R}_\text{NS} \]
  \begin{itemize}
  \item $\ddot{R}_\text{2B}$ uses point-mass equations
    \begin{itemize}
    \item Works for points
    \item Works for spheres (shell theorem)
    \end{itemize}
  \item Earth is neither
    \begin{itemize}
    \item Structural rigidity
    \item Centrifugal forces
    \item Tidal forces
    \end{itemize}
  \end{itemize}
\end{frame}

\begin{frame}{Perturbation: Indirect Oblation}
  \[ \ddot{R} = \ddot{R}_\text{2B} + \ddot{R}_\text{PM} + \ddot{R}_\text{NS}  + \ddot{R}_\text{IO}\]
  \begin{itemize}
  \item Earth is not an inertial reference frame
    \begin{itemize}
    \item Has its own orbit around Sol
    \item Yanked around by Luna inside that orbit
    \end{itemize}
  \item ``Shaky Camera'' effect
  \end{itemize}
\end{frame}

\begin{frame}{Perturbation: Drag}
  \[ \ddot{R} = \ddot{R}_\text{2B} + \ddot{R}_\text{PM} + \ddot{R}_\text{NS}  + \ddot{R}_\text{IO} + \ddot{R}_\text{D} \]

  \begin{itemize}
  \item Drag equation: $F_\text{D} = \frac{1}{2}\rho v^2 C_\text{D} A$
  \item Scales by
    \begin{itemize}
    \item Object shape and orientation ($C_\text{D}, A$).
    \item Square of object velocity $v^2$.
    \item Atmospheric density $rho$, drops ~exponentially with altitude.
    \end{itemize}
  \item LEO objects (altitude < 2000 km) are low and fast
    \begin{itemize}
    \item experience non-negligible drag
    \end{itemize}
  \item Bonus: non-periodic and non-conservative.
  \end{itemize}
\end{frame}

\begin{frame}{Perturbation: Solar Radiation Pressure}
  \[ \ddot{R} = \ddot{R}_\text{2B} + \ddot{R}_\text{PM} + \ddot{R}_\text{NS}  + \ddot{R}_\text{IO} + \ddot{R}_\text{D} + \ddot{R}_\text{SRP}\]

  \begin{itemize}
  \item $\gamma := \sqrt{\frac{c^2}{c^2 - v^2}}$
  \item $ p = \gamma m v$
  \item photons: $m\rightarrow0, \gamma \rightarrow \infty $
    \begin{itemize}
    \item $p \rightarrow ?$
    \item God's math: $ p = \frac{h}{\lambda} $
    \end{itemize}
  \item Absorbing and emitting light imparts momentum
    \begin{itemize}
    \item Sunlight never stops: SRP
    \item Most impactful on higher altitude orbits
    \item Non-periodic and non-conservative
    \end{itemize}
  \end{itemize}
\end{frame}

\begin{frame}{Perturbation: Thrust}
  \[ \ddot{R} = \ddot{R}_\text{2B} + \ddot{R}_\text{PM} + \ddot{R}_\text{NS}  + \ddot{R}_\text{IO} + \ddot{R}_\text{D} + \ddot{R}_\text{SRP} + \ddot{R}_\text{T} \]

  \begin{itemize}
  \item Orbital payloads commonly come equipped with maneuvering thrusters
    \begin{itemize}
    \item Chemical burns (fast, short)
    \item Electric propulsion (slow, long)
    \end{itemize}
  \item Good news: allows for doing something about predicted collisions
  \item Bad news: Non-periodic, non-conservative, AND non-physical(-ish)
  \end{itemize}
\end{frame}

\begin{frame}{Perturbation Impacts}
  \begin{itemize}
  \item Low-to-medium fidelity diff-eqs can be solved analytically
    \begin{itemize}
      \item E.g. Brouwer models, SGP4/SDP4
    \end{itemize}
  \item High-fidelity generally resort to numerical integration
    \begin{itemize}
      \item E.g. NORAD Special Perturbations (SP)
    \end{itemize}
  \item Low- and high-fidelity models can diverge significantly, rapidly
    \begin{itemize}
    \item By kilometers
    \item Within a few orbital periods (i.e. hours)
    \end{itemize}
  \end{itemize}
\end{frame}

\subsection{Consequence}

\begin{frame}{Mission Safety}
  \[ E(\text{Cost}(X)) = P(X) \cdot \text{Cost}(X) \]
  
  \begin{itemize}
  \item Plausible $\text{Cost}(X)$: 100 million USD
  \item Plausible $P(X)$: 2e-4
  \item $E(\text{Cost}(X)) = 2 \cdot 10^{-4} \cdot 10^8 = 20000$ USD
    \begin{itemize}
    \item Might be worth mitigating
    \item Although, $\sim 85\%$ of likely-lethal conjunctors aren't even
      tracked \ldots
    \end{itemize}
  \end{itemize}
\end{frame}

\begin{frame}{Domain Safety I}

  \begin{itemize}
  \item Orbit contention is self-reinforcing
    \begin{itemize}
    \item More objects means more conjunctions
    \item More conjunctions means more collisions
    \item More collisions means more objects
    \end{itemize}
  \item Critical density $\rightarrow$ runaway, sustained fragmentation
    \begin{itemize}
    \item Kessler syndrome
    \end{itemize}
  \item Sub-critical density increases still increase hazard to ecosystem
  \end{itemize}
\end{frame}

\begin{frame}{Domain Safety II}
  \begin{itemize}
  \item Vested public interest in controlling flux
  \item Must avoid collisions, especially between large objects
    \begin{itemize}
    \item Debris potential strongly linked to object size
    \item Largest objects are best tracked
    \item Objects follow a power-law distribution: many, many small pieces,
      comparatively few large
    \end{itemize}
  \end{itemize}
\end{frame}

\section{Conjunction Identification}
\subsection{Volumetric Screening}

\begin{frame}{Conjunction Screening}
  \begin{itemize}
  \item Before a conjunction can be analyzed it must be identified
    \begin{itemize}
    \item Want to know the time of closest approach (TCA)
    \item and of course the conjunctors' states at TCA
    \end{itemize}
  \item Technically not a CARA responsibility
    \begin{itemize}
    \item Screening computed for CARA by CSpOC
    \item Uses a method called Computation of Miss Between Orbits (COMBO)
    \end{itemize}
  \end{itemize}
\end{frame}

\begin{frame}{COMBO in Pictures}
  % TODO: create a flying ellipsoid image-set.
\end{frame}

\begin{frame}{COMBO in Pseudocode}
  \begin{algorithmic}
    \Procedure{COMBO}{}
    \For{$p, s, t \in \textit{Primaries} \times \textit{Secondaries} \times \textit{Time Slices}$}
      \State $V \gets \text{ellipse around }p(t)$
      \If{$s(t).\text{pos} \notin V$}
        \State \textbf{continue}
      \EndIf
      \State $d(\tau) := || p(\tau).\text{pos}  - s(\tau).\text{pos} ||_2$
      \If{$\text{sign}(d'(t)) = \text{sign}(d'(t+\Delta t))$}
        \State \textbf{continue}
      \EndIf
      \State $t^* \gets \operatorname*{argmin}_{[t, t + \Delta t]} d$
      \State $\text{Emit } (p, s, t^*)$
    \EndFor
    \EndProcedure  
  \end{algorithmic}
\end{frame}

\section{Conjunction Analysis}
\subsection{Risk Measures}

\begin{frame}{Standoff Distance}
  \begin{itemize}
  \item Most intuitive measure of safety
    \begin{itemize}
    \item If we are far apart, of course we aren't touching.
    \item Implicitly part of volumetric screening regimes.
    \end{itemize}

  \item Difficult to map distance onto risk.
    \begin{itemize}
    \item How far apart is far? Meters? Kilometers?
    \end{itemize}
    
  \item Cannot capture uncertainty
    \begin{itemize}
    \item What if our measurements of a satellite's state are known to be imprecise?
    \end{itemize}
    
  \item Tendency toward conservatism
    \begin{itemize}
      \item sometimes desirable, e.g. around human space flight assets
    \end{itemize}
  \end{itemize}
\end{frame}

\begin{frame}{Probability of Collision}
  \begin{itemize}
  \item Most commonly used measure of safety.

  \item Answers the challenges with standoff distance
    \begin{itemize}
    \item Maps naturally to risk.
    \item Captures and describes uncertainty and imprecision.
    \item Allows for mindfully tuned risk postures.
    \end{itemize}

  \item But! Can suffer from probability dilution.
    \begin{itemize}
    \item Space is big. Really big. Really, really big.
    \item Rubbish measurements $\Rightarrow P_C \approx 0$.
    \item Probability is ``diluted'' across space.
    \end{itemize}
  \end{itemize}
\end{frame}

\begin{frame}{Severity Estimation}
  \begin{itemize}
  \item For the domain:
    \begin{itemize}
    \item Different collisions create different amounts of debris.
    \item EVOLVE: empirically determined model mapping relative masses and
      velocities to expected fragmentation count.
    \item Basic rules: fast is bad, big is bad.
    \item Of course, ultimately dependent on O/Os to maneuver/mitigate.
    \end{itemize}

  \item For O/Os:
    \begin{itemize}
    \item Everything is moving too damned fast
    \item Everything is too damned expensive
    \item A hit is a hit is a hit is a bad time
    \end{itemize}
  \end{itemize}
\end{frame}

\subsection{2D $P_C$}
% TODO: describe the 2D P_C algorithm

\subsection{3D $P_C$}
% TODO: describe 3D P_C algorithm

\subsection{Monte Carlo}
% TODO: describe from-epoch and from-TCA monte carlo methods.

%% Q&A and Further reading.
\miniframesoff
\section*{}
\begin{frame}{References}[allowframebreaks]
  \tiny
% TODO
%    \printbibliography
\end{frame}

\begin{frame}{Self Link}
  The source for this presentation is hosted at
\url{https://github.com/alan-christopher/cara-edu}.

\end{frame}
\end{document}
